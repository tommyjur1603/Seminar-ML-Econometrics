% Options for packages loaded elsewhere
\PassOptionsToPackage{unicode}{hyperref}
\PassOptionsToPackage{hyphens}{url}
%
\documentclass[
  12pt,
  a4paper,
  oneside]{article}

\usepackage{amsmath,amssymb}
\usepackage{setspace}
\usepackage{iftex}
\ifPDFTeX
  \usepackage[T1]{fontenc}
  \usepackage[utf8]{inputenc}
  \usepackage{textcomp} % provide euro and other symbols
\else % if luatex or xetex
  \usepackage{unicode-math}
  \defaultfontfeatures{Scale=MatchLowercase}
  \defaultfontfeatures[\rmfamily]{Ligatures=TeX,Scale=1}
\fi
\usepackage{lmodern}
\ifPDFTeX\else  
    % xetex/luatex font selection
  \setmainfont[]{Times New Roman}
  \setsansfont[]{Times New Roman}
\fi
% Use upquote if available, for straight quotes in verbatim environments
\IfFileExists{upquote.sty}{\usepackage{upquote}}{}
\IfFileExists{microtype.sty}{% use microtype if available
  \usepackage[]{microtype}
  \UseMicrotypeSet[protrusion]{basicmath} % disable protrusion for tt fonts
}{}
\makeatletter
\@ifundefined{KOMAClassName}{% if non-KOMA class
  \IfFileExists{parskip.sty}{%
    \usepackage{parskip}
  }{% else
    \setlength{\parindent}{0pt}
    \setlength{\parskip}{6pt plus 2pt minus 1pt}}
}{% if KOMA class
  \KOMAoptions{parskip=half}}
\makeatother
\usepackage{xcolor}
\usepackage[top=30mm,bottom=30mm,left=25mm,right=25mm]{geometry}
\setlength{\emergencystretch}{3em} % prevent overfull lines
\setcounter{secnumdepth}{5}
% Make \paragraph and \subparagraph free-standing
\ifx\paragraph\undefined\else
  \let\oldparagraph\paragraph
  \renewcommand{\paragraph}[1]{\oldparagraph{#1}\mbox{}}
\fi
\ifx\subparagraph\undefined\else
  \let\oldsubparagraph\subparagraph
  \renewcommand{\subparagraph}[1]{\oldsubparagraph{#1}\mbox{}}
\fi


\providecommand{\tightlist}{%
  \setlength{\itemsep}{0pt}\setlength{\parskip}{0pt}}\usepackage{longtable,booktabs,array}
\usepackage{calc} % for calculating minipage widths
% Correct order of tables after \paragraph or \subparagraph
\usepackage{etoolbox}
\makeatletter
\patchcmd\longtable{\par}{\if@noskipsec\mbox{}\fi\par}{}{}
\makeatother
% Allow footnotes in longtable head/foot
\IfFileExists{footnotehyper.sty}{\usepackage{footnotehyper}}{\usepackage{footnote}}
\makesavenoteenv{longtable}
\usepackage{graphicx}
\makeatletter
\def\maxwidth{\ifdim\Gin@nat@width>\linewidth\linewidth\else\Gin@nat@width\fi}
\def\maxheight{\ifdim\Gin@nat@height>\textheight\textheight\else\Gin@nat@height\fi}
\makeatother
% Scale images if necessary, so that they will not overflow the page
% margins by default, and it is still possible to overwrite the defaults
% using explicit options in \includegraphics[width, height, ...]{}
\setkeys{Gin}{width=\maxwidth,height=\maxheight,keepaspectratio}
% Set default figure placement to htbp
\makeatletter
\def\fps@figure{htbp}
\makeatother
% definitions for citeproc citations
\NewDocumentCommand\citeproctext{}{}
\NewDocumentCommand\citeproc{mm}{%
  \begingroup\def\citeproctext{#2}\cite{#1}\endgroup}
\makeatletter
 % allow citations to break across lines
 \let\@cite@ofmt\@firstofone
 % avoid brackets around text for \cite:
 \def\@biblabel#1{}
 \def\@cite#1#2{{#1\if@tempswa , #2\fi}}
\makeatother
\newlength{\cslhangindent}
\setlength{\cslhangindent}{1.5em}
\newlength{\csllabelwidth}
\setlength{\csllabelwidth}{3em}
\newenvironment{CSLReferences}[2] % #1 hanging-indent, #2 entry-spacing
 {\begin{list}{}{%
  \setlength{\itemindent}{0pt}
  \setlength{\leftmargin}{0pt}
  \setlength{\parsep}{0pt}
  % turn on hanging indent if param 1 is 1
  \ifodd #1
   \setlength{\leftmargin}{\cslhangindent}
   \setlength{\itemindent}{-1\cslhangindent}
  \fi
  % set entry spacing
  \setlength{\itemsep}{#2\baselineskip}}}
 {\end{list}}
\usepackage{calc}
\newcommand{\CSLBlock}[1]{\hfill\break\parbox[t]{\linewidth}{\strut\ignorespaces#1\strut}}
\newcommand{\CSLLeftMargin}[1]{\parbox[t]{\csllabelwidth}{\strut#1\strut}}
\newcommand{\CSLRightInline}[1]{\parbox[t]{\linewidth - \csllabelwidth}{\strut#1\strut}}
\newcommand{\CSLIndent}[1]{\hspace{\cslhangindent}#1}

\usepackage{fancyhdr}
\pagestyle{fancy}
\makeatletter
\@ifpackageloaded{caption}{}{\usepackage{caption}}
\AtBeginDocument{%
\ifdefined\contentsname
  \renewcommand*\contentsname{Table of contents}
\else
  \newcommand\contentsname{Table of contents}
\fi
\ifdefined\listfigurename
  \renewcommand*\listfigurename{List of Figures}
\else
  \newcommand\listfigurename{List of Figures}
\fi
\ifdefined\listtablename
  \renewcommand*\listtablename{List of Tables}
\else
  \newcommand\listtablename{List of Tables}
\fi
\ifdefined\figurename
  \renewcommand*\figurename{Figure}
\else
  \newcommand\figurename{Figure}
\fi
\ifdefined\tablename
  \renewcommand*\tablename{Table}
\else
  \newcommand\tablename{Table}
\fi
}
\@ifpackageloaded{float}{}{\usepackage{float}}
\floatstyle{ruled}
\@ifundefined{c@chapter}{\newfloat{codelisting}{h}{lop}}{\newfloat{codelisting}{h}{lop}[chapter]}
\floatname{codelisting}{Listing}
\newcommand*\listoflistings{\listof{codelisting}{List of Listings}}
\makeatother
\makeatletter
\makeatother
\makeatletter
\@ifpackageloaded{caption}{}{\usepackage{caption}}
\@ifpackageloaded{subcaption}{}{\usepackage{subcaption}}
\makeatother
\ifLuaTeX
  \usepackage{selnolig}  % disable illegal ligatures
\fi
\usepackage{bookmark}

\IfFileExists{xurl.sty}{\usepackage{xurl}}{} % add URL line breaks if available
\urlstyle{same} % disable monospaced font for URLs
\hypersetup{
  pdftitle={Seminar Term Paper},
  pdfauthor={Thomas Jürgensen},
  hidelinks,
  pdfcreator={LaTeX via pandoc}}

\title{Seminar Term Paper}
\author{Thomas Jürgensen}
\date{Invalid Date}

\begin{document}
\begin{titlepage}
\begin{center}

\Large
SEMINAR TERM PAPER

\vspace{0.5cm}
\Large
LUDWIG-MAXIMILIANS-UNIVERSITÄT MÜNCHEN\\
DEPARTMENT OF STATISTICS

\vspace{0.5cm}

\rule{\textwidth}{1.5pt}
\LARGE
\textbf{Estimation and Inference of Heterogeneous Treatment Effects using Random Forests}
\rule{\textwidth}{1.5pt}

\vspace{0.5cm}

\Large
\textbf{Thomas Jürgensen}\\
\large
Matriculation Nr.: 12369869

\vspace{0.5cm}

Supervised by Dr. Daniel Wilhelm

\vspace{0.5cm}

\large
Munich, August 31\textsuperscript{st}, 2025
      
\vspace{0.5cm}

%\includegraphics[width = 0.4\textwidth]{sigillum.png}

%\vspace{0.5cm}

\vfill

\end{center}
\end{titlepage}
\newpage
\newpage
\renewcommand*\contentsname{Contents}
\setcounter{tocdepth}{3}
\renewcommand*\listfigurename{List of figures}
\renewcommand*\listtablename{List of tables}\setstretch{1.5}
\pagenumbering{roman}

\tableofcontents

\newpage

\section{Introduction}\label{sec-intro}

\pagenumbering{arabic}

One of the most important tasks in economics, medicine, statistics, and
many other disciplines is estimating the causal effect of a treatment or
intervention. Assuming that a treatment's impact is largely consistent
across people and observations, a large portion of the literature has
historically concentrated on estimating average treatment effects
(ATEs). However, on many different scenarios, the effects of some sort
of treatment are intrinsically heterogeneous, which means that they
differ greatly among different individuals, observations or sub-groups.
For instance, a newly developed medical intervention might be effective
for younger patients but ineffective for elderly patients, or its effect
might be different for men than it is for women. Similarly, an
educational program may have little effect on high achievers but improve
results for students with inadequate prior preparation.

\qquad The development of adaptable techniques that can reveal
individual differences in treatment response has been spurred by the
growing interest in these so-called
\textbf{heterogeneous treatment effects}. However, there are a number of
difficulties in precisely estimating them. In high-dimensional settings
or when the form of heterogeneity is complex, traditional statistical
techniques like regression with interaction terms or subgroup analysis
can lose their reliability. Moreover, a lot of machine learning
techniques are very good at predicting results, but they are not made
for drawing conclusions about causality, and they usually don't have
reliable instruments for measuring uncertainty.

\qquad In response to this methodological gap, Wager and Athey (2018)
propose the \textbf{causal forest}, an adaptation of Breiman's random
forest algorithm (Breiman 2001) tailored for the estimation of
\textbf{conditional
average treatment effects} (or CATEs). Their approach not only offers
the flexibility of nonparametric machine learning but also provides a
theoretical framework for statistical inference, including pointwise
confidence intervals for individual treatment effects. This is made
possible through key innovations such as the use of \textbf{honest}
trees, which separate the data used for tree construction from the data
used for estimation, and a consistent variance estimation technique
based on the infinitesimal jackknife for random forests developed by
Wager, Hastie, and Efron (2013). The causal forest method thus
represents a significant advancement in the field of causal inference.
It bridges the gap between machine learning's ability to capture complex
relationships and the statistical rigor needed for credible statistical
inference.

\qquad The remainder of this term paper will provide a detailed overview
of Wager and Athey's methodology, the theoretical guarantees supporting
causal forests, empirical performance as demonstrated in an empirical
example, and a discussion of the method's limitations and potential
extensions.

\subsection{Literature Review}\label{literature-review}

Following the development of causal forests, a number of studies have
explored their application, refinement, and theoretical foundations.
Athey and Wager (2019) applied the method to education data,
highlighting practical considerations such as accounting for clustered
errors and discussing how causal forests use propensity scores to be
more robust to confounding. Davis and Heller (2017) demonstrated its
usefulness in randomized trials for youth employment programs, using
estimated CATEs to identify subgroups with the largest responses to the
intervention, and Lechner (2018) proposed modifications to extend the
causal forest framework to models involving multiple treatments,
enhancing flexibility in policy evaluation and stratified causal effect
estimation.

\qquad Further theoretical and applied developments have broadened the
scope of the method. Gavrilova, Langørgen, and Zoutman (2025) developed
a difference-in-differences causal forest, which provide consistent
estimates with a parallel trend assumption. The methodology has also
begun to see broader uptake in clinical and epidemiological research:
Susukida et al. (2024) used causal forests to explore heterogeneous
treatment effects in psychosocial interventions for substance use
disorder, revealing nuanced findings like limited overall heterogeneity
but potential subgroup-specific effects.

\qquad A particularly important recent contribution was made by
Cattaneo, Klusowski, and Tian (2022), who offer a rigorous theoretical
perspective on recursive partitioning methods for pointwise inference.
They demonstrate that adaptive (non-honest) trees may fail to achieve
even polynomial convergence rates and can be unreliable for inference,
whereas random forests transform weak base learners into estimators with
near optimal convergence properties through subsampling and random
feature selection. Their results provide strong theoretical
justification that supports the theoretical necessity of honesty,
subsampling, and randomness for pointwise valid causal inference via
forest methods.

\section{Methodology Overview}\label{methodology-overview}

Following the motivation outlined in Section~\ref{sec-intro}, this
second chapter introduces the methodology developed by Wager and Athey
(2018): the \textbf{causal forest}. This method builds upon the
traditional machine learning technique, the random forest, in order to
estimate heterogeneous treatment effects in a flexible and statistically
principled way.

\subsection{Defining Causal Forests}\label{sec-causal-forests}

\subsubsection{Estimating Conditional Average Treatment
Effects}\label{estimating-conditional-average-treatment-effects}

Estimating the Conditional Average Treatment Effect (CATE), which is
defined as \begin{equation} \label{eq:1}
\tau(x) = \mathbb{E}[Y(1) - Y(0) | X = x],
\end{equation} is a common goal in causal inference. Here, \(Y(1)\) and
\(Y(0)\) represent the potential outcomes under treatment and control,
respectively, and \(X\) is a vector of observed covariates. The
objective is to estimate the expected causal effect of receiving the
treatment for any individuals with a given set of characteristics
\(X = x\). However, estimating \(\tau(x)\) is inherently challenging
because only one of the two potential outcomes is observed for each
individual, since each person can only be either treated or not treated.
To make the estimation feasible, the standard assumptions of
\textbf{unconfoundedness} and \textbf{overlap} are imposed:

\begin{itemize}
\item
  \textbf{Unconfoundedness}: According to this assumption, the treatment
  assignment is essentially random, conditional on the covariates \(X\),
  and is independent of the potential outcomes, i.e.,
  \(Y(1), Y(0) \perp W | X\), meaning that treatment assignment \(W\),
  conditional on covariates \(X\), is essentially random.
\item
  \textbf{Overlap}: For every covariate \(x\), both treated and control
  units are observed because the probability of receiving treatment is
  positive for all values of \(X\) but not equal to 1, i.e.,
  \(0 < \mathbb{P}(W = 1 | X = x) < 1\) for all \(x\), ensuring that
  both treated and control units are observed for every covariate \(x\).
\end{itemize}

The CATE \(\tau(x)\) can now be estimated from the observed data because
of these assumptions, which enable us to treat observational data as
though it had originated from a randomized experiment, conditional on
covariates.

\subsubsection{Causal Trees}\label{causal-trees}

The decision tree, a popular and extensively used machine learning
algorithm that recursively splits the covariate space into disjoint
regions (also called ``leaves''), within which a basic model is applied,
is a logical place to start. Conventional regression trees are designed
to forecast outcomes rather than estimate causal effects. Causal trees,
on the other hand, are intended to directly estimate treatment effects.
The difference in average outcomes between treated and control units is
used to estimate the treatment effect in each leaf:
\begin{equation} \label{eq:2}
\hat{\tau}(x) = \left( \frac{1}{n_1} \sum_{i: W_i = 1, X_i \in L} Y_i \right) - \left( \frac{1}{n_0} \sum_{i: W_i = 0, X_i \in L} Y_i \right),
\end{equation} where \(L\) is the leaf that contains the covariate
\(x\), \(n_1\) is the number of treated units in leaf \(L\), and \(n_0\)
is the number of control units in leaf \(L\). It is crucial to note
that, because of the unconfoundedness assumption, we can simply
calculate the difference in average outcomes between treated and control
units, and this difference would recover a causal effect. Without this
assumption, the difference in observed outcomes would not necessarily
reflect the true causal effect, as there could be confounding factors
that influence both treatment assignment and outcomes.

\qquad Although causal trees offer flexible estimates of the localized
effects of treatments, a single tree is highly sensitive to data
perturbations and prone to high variance, as is well known from
traditional machine learning. The causal forest algorithm is based on an
ensemble of trees that are used to assess this problem.

\subsubsection{Causal Forests}\label{causal-forests}

A causal forest is an ensemble of many causal trees, each built on a
random subsample of the data. By aggregating the treatment effect
estimates from many such trees, causal forests reduce variance and
produce more stable treatment effect estimates. Formally, the causal
forest estimator is given by: \begin{equation} \label{eq:3}
\hat{\tau}(x) = \frac{1}{B} \sum_{b=1}^{B} \hat{\tau}_b(x),
\end{equation} where \(\hat{\tau}_b(x)\) is the treatment effect
estimate from the \(b\)-th causal tree, and \(B\) is the total number of
trees grown in the forest.

\qquad However, an important issue arises in the construction of causal
trees and forests. If the same data is used both to determine the tree
structure (i.e., the locations of splits) and to estimate treatment
effects within the resulting leaf nodes, the resulting estimates may
suffer from overfitting. This is especially problematic for inference,
since overfitting would lead to biased treatment effect estimates. More
critically, this kind of bias would undermine the validity of any
subsequent statistical inference, resulting in invalid and unreliable
confidence intervals.

\subsubsection{Honest Trees}\label{honest-trees}

To address this problem, Wager and Athey (2018) introduce the concept of
\textbf{honest} trees, a design choice that separates the tasks of model
selection and estimation.

\textbf{Definition 1 (Honest Trees)}\footnote{This definition is adapted
  from Wager and Athey (2018).}:
\textit{A tree is called \textbf{honest} if it uses separate subsamples for two distinct purposes:}

\begin{itemize}
\item
  \textit{One subsample is used to
  determine the tree
  structure, that is, where
  to place the splits,}
\item
  \textit{The other is used to
  estimate the treatment
  effects within each leaf.}
\end{itemize}

This separation prevents the model from overfitting during tree
construction, ensuring that the treatment effect estimates remain
unbiased.

\qquad Honesty can be implemented in various ways. However, in causal
forests, Wager and Athey (2018) suggest the implementation of honest
trees through the so-called ``double-sample'' approach, where each
subsample used to grow a causal tree is split into two halves: one half
for splitting and tree growing, one the other half for the estimation of
treatment effects.

\subsection{Asymptotic Theory}\label{asymptotic-theory}

As we have seen in Section~\ref{sec-causal-forests}, causal forests
provide a very flexible and nonparametric method for estimating
heterogeneous treatment effects. However, the main theoretical
contribution of causal forests is that they permit valid statistical
inference. The well-developed asymptotic theory of causal forests
ensures both consistency and asymptotic normality of the treatment
effect estimates under appropriate conditions and assumptions, in
contrast to the majority of machine learning techniques, which are
mainly optimized for prediction and do not provide uncertainty
quantification. The main theoretical findings in favor of using causal
forests for statistical inference are presented in this chapter.

\subsubsection{Consistency of the
Estimator}\label{consistency-of-the-estimator}

The first fundamental result of Wager and Athey (2018) is
\textbf{pointwise consistency}. This result guarantees that, as the
sample size \(n\) increases to infinity, the treatment effect estimator
\(\hat{\tau}(x)\) converges in probability to the true conditional
average treatment effect \(\tau(x)\) for any fixed covariate
\(x \in X\), i.e., \begin{equation} \label{eq:4}
\hat{\tau}(x) \xrightarrow{p} \tau(x) \quad \text{as} \quad n \to \infty.
\end{equation} This result is very important because it ensures that the
causal forest estimator will recover the correct treatment effects at
each data point \(x\) as the sample size grows, making it a reliable
tool for statistical inference. To achieve pointwise consistency, one
additional assumption is needed, and that is that both conditional mean
functions \(\mathbb{E}[ Y(0) | X = x ]\) and
\(\mathbb{E}[ Y(1) | X = x ]\) are Lipschnitz-continuous.

\subsubsection{Asymptotic Normality}\label{asymptotic-normality}

The second fundamental result of Wager and Athey (2018) is
\textbf{asymptotic normality}. This result states that, under certain
conditions, the treatment effect estimator \(\hat{\tau}(x)\) is
asymptotically normally distributed around the true treatment effect
\(\tau(x)\).

\textbf{Theorem 1 (Asymptotic Normality)}\footnote{This theorem is
  adapted from Wager and Athey (2018).}:

\textit{Let $(\textbf{X}_i, Y_i, W_i)_{i=1}^n$ be $n$ i.i.d. training examples, where:}

\begin{itemize}
\item
  \textit{$\textbf{X}_i \in [0,1]^d$ are the covariates,}
\item
  \textit{$Y_i \in \mathbb{R}$ is the observed outcomes,}
\item
  \textit{$W_i \in \{0,1\}$ indicates the binary treatment assignment.}
\end{itemize}

\textit{Suppose that these training examples satisfy the following conditions:}

\begin{itemize}
\item
  \textit{The treatment assignment $W_i$ is unconfounded and has overlap,}
\item
  \textit{The conditional means $\mathbb{E}[ Y(0) | \textbf{X} = x ]$ and $\mathbb{E}[ Y(1) | \textbf{X} = x ]$ are Lipschnitz-continuous,}
\item
  \textit{The conditional variance is bounded, i.e.}
  \(\text{sup}_x \text{Var} (Y | \textbf{X} = x) < \infty\).
\item
  \textit{The covariates are \textbf{independent} and \textbf{uniformly distributed}, i.e. $\textbf{X}_i \sim \mathcal{U}([0,1]^d)$.}
\end{itemize}

\textit{Given these conditions, let $\Gamma$ be an \textbf{honest} causal forest, where:}

\begin{itemize}
\item
  \textit{Each causal tree is built on a random subsample of size $s_n \propto n^\beta$ for some $\beta_{min} < \beta < 1$, where $\beta_{min}$ depends on covariate dimension $d$ and regularity parameter $\alpha$,}
\item
  \textit{each causal tree is $𝛂$\textbf{-regular}, meaning that every split sends at least an $\alpha$-fraction of the subsample to each child node (in this case, $\alpha \le 0.2$ is used),}
\end{itemize}

\textit{Then, for any fixed covariate $x \in [0,1]^d$, the treatment effect estimator $\hat{\tau}(x)$ is \textbf{asymptotically normal} and \textbf{centered}. That is:}
\begin{equation} \label{eq:5}
\frac{\left(\hat{\tau}(x) - \tau(x)\right)}{\sqrt{\text{Var}\left(\hat{\tau}(x)\right)}} \xrightarrow{d} \mathcal{N}(0,1) \quad \text{as} \quad n \to \infty.
\end{equation}
\textit{Furthermore, the infinitesimal jackknife}\footnote{The
  infinitesimal jackknife is the name of a variance estimation method
  for random forests developed by Efron (2014) and Wager, Hastie, and
  Efron (2013).} \textit{(IJ) variance estimator, which is defined as:}
\begin{equation} \label{eq:6}
\hat{V}_{IJ}(x) = \frac{n-1}{n}\left(\frac{n}{n-s}\right)^2 \sum_{i=1}^{n} \left(\text{Cov}_{\ast}\left[\hat{\tau}^{\ast}_{b}(x), N_{ib}^{\ast}\right]\right)^2,
\end{equation}
\textit{where $\hat{\tau}^{\ast}_b(x)$ is the treatment effect estimate given by the $b$-th tree, and $N_{ib}^{\ast} \in \{0,1\}$ indicates whether the training example $i$ was used for the $b$-th tree, is a consistent estimator of the variance, in the sense that:}
\begin{equation} \label{eq:7}
\frac{\hat{V}_{IJ}(x)}{\text{Var}(\hat{\tau}(x))} \xrightarrow{p} 1
\quad \text{as} \quad n \to \infty.
\end{equation}

\qquad This result establishes causal forests as a method suitable not
only for the flexible, nonparametric estimation of heterogeneous
treatment effects, but also for conducting
\textbf{asymptotically valid statistical inference}. The asymptotic
normality of the estimator enables the construction of confidence
intervals around the estimated treatment effect at each point \(x\),
thereby allowing researchers to quantify uncertainty in a principled
way, i.e. \begin{equation} \label{eq:8}
\hat{\tau}(x) \pm z_{1 - \frac{\alpha}{2}} \cdot \sqrt{\hat{V}_{IJ}(x)},
\end{equation} where \(z_{\alpha/2}\) is the critical value from the
standard normal distribution corresponding to the desired confidence
level.

\qquad However, the conditions and assumptions that underlie the causal
forest construction in \textbf{Theorem 1} are crucial to the asymptotic
normality result. For instance, the estimator may become biased if the
honesty condition is broken, which would mean that the same data is used
to estimate treatment effects within leaves as well as to choose tree
splits. Confidence intervals may become invalid as a result of this
bias, which compromises the central limit theorem that underpins sound
inference. Consequently, the method's theoretical guarantees, such as
consistency and reliable statistical inference, might no longer be valid
in the absence of these structural safeguards.

\section{Empirical Analysis}\label{empirical-analysis}

\section{Discussion}\label{discussion}

\section{Conclusion}\label{conclusion}

\appendix

\section{\texorpdfstring{Appendix
\label{appendixa}}{Appendix }}\label{appendix}

\section*{References}\label{references}
\addcontentsline{toc}{section}{References}

\pagestyle{plain}

\phantomsection\label{refs}
\begin{CSLReferences}{1}{0}
\bibitem[\citeproctext]{ref-athey2019}
Athey, Susan, and Stefan Wager. 2019. {``Estimating Treatment Effects
with Causal Forests: An Application.''}
\url{https://doi.org/10.48550/ARXIV.1902.07409}.

\bibitem[\citeproctext]{ref-breiman2001}
Breiman, Leo. 2001. \emph{Machine Learning} 45 (1): 5--32.
\url{https://doi.org/10.1023/a:1010933404324}.

\bibitem[\citeproctext]{ref-cattaneo2022}
Cattaneo, Matias D., Jason M. Klusowski, and Peter M. Tian. 2022. {``On
the Pointwise Behavior of Recursive Partitioning and Its Implications
for Heterogeneous Causal Effect Estimation.''}
\url{https://doi.org/10.48550/ARXIV.2211.10805}.

\bibitem[\citeproctext]{ref-davis2017}
Davis, Jonathan M. V., and Sara B. Heller. 2017. {``Using Causal Forests
to Predict Treatment Heterogeneity: An Application to Summer Jobs.''}
\emph{American Economic Review} 107 (5): 546--50.
\url{https://doi.org/10.1257/aer.p20171000}.

\bibitem[\citeproctext]{ref-efron2014}
Efron, Bradley. 2014. {``Estimation and Accuracy After Model
Selection.''} \emph{Journal of the American Statistical Association} 109
(507): 991--1007. \url{https://doi.org/10.1080/01621459.2013.823775}.

\bibitem[\citeproctext]{ref-gavrilova2025}
Gavrilova, Evelina, Audun Langørgen, and Floris T. Zoutman. 2025.
{``Difference{-}in{-}Difference Causal Forests With an Application to
Payroll Tax Incidence in Norway.''} \emph{Journal of Applied
Econometrics}, July. \url{https://doi.org/10.1002/jae.70001}.

\bibitem[\citeproctext]{ref-lechner2018}
Lechner, Michael. 2018. {``Modified Causal Forests for Estimating
Heterogeneous Causal Effects.''}
\url{https://doi.org/10.48550/ARXIV.1812.09487}.

\bibitem[\citeproctext]{ref-susukida2024}
Susukida, Ryoko, Masoumeh Amin-Esmaeili, Elena Badillo-Goicoechea, Trang
Q. Nguyen, Elizabeth A. Stuart, Michael Rosenblum, Kelly E. Dunn, and
Ramin Mojtabai. 2024. {``Application of Causal Forest Model to Examine
Treatment Effect Heterogeneity in Substance Use Disorder Psychosocial
Treatments.''} \emph{International Journal of Methods in Psychiatric
Research} 34 (1). \url{https://doi.org/10.1002/mpr.70011}.

\bibitem[\citeproctext]{ref-wager2018}
Wager, Stefan, and Susan Athey. 2018. {``Estimation and Inference of
Heterogeneous Treatment Effects Using Random Forests.''} \emph{Journal
of the American Statistical Association} 113 (523): 1228--42.
\url{https://doi.org/10.1080/01621459.2017.1319839}.

\bibitem[\citeproctext]{ref-wager2013}
Wager, Stefan, Trevor Hastie, and Bradley Efron. 2013. {``Confidence
Intervals for Random Forests: The Jackknife and the Infinitesimal
Jackknife.''} \url{https://doi.org/10.48550/ARXIV.1311.4555}.

\end{CSLReferences}



\end{document}
